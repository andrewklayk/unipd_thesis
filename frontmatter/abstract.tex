%!TEX root = ../dissertation.tex
As machine-learning algorithms proliferate, there are growing concerns regarding their fairness. Can we stop AI, trained on real-world data, from reproducing and exacerbating real-world biases? There exists a growing body of work on fairness in AI, but often focussing rather narrowly on classification problems. Online advertising and job-candidate rankings, for example, utilize ranking algorithms instead of classification algorithms. The objectives of this work are to (a) describe the theoretical basis of fairness in ranking and the metrics used to evaluate it, (b) explore several existing post-processing algorithms for fairness in ranking, (c) introduce a novel randomized algorithm based on Mallows distribution that offers a tradeoff between fairness and accuracy, and (d) compare its performance in terms of obtained fairness and loss of ranking accuracy to such of existing deterministic algorithms. 